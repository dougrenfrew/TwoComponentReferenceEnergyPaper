\section{Introduction}
\subsection{Canonical and noncanonical design are useful}
The computational design of macromolecules like proteins has permitted dramatic advances in chemical and synthetic biology.
Protein design has resulted in advances in basic knowledge, such as the creation of proteins with novel folds[ref], as well as industrial and medical applications, such as the creation and optimization of enzymes that can conduct reactions that are not performed by natural enzymes[ref], as well as <some example of protein design for medicine or perhaps synthetic biology>[ref].

In recent years, the potential to extend existing peptide design methods to chemically diverse groups of heteropolymer molecules which display peptide-like properties such as secondary and tertiary structure[ref?]. Examples include peptoids[ref], oligo-oxopiperazines(spelling?)[ref], beta-peptides[ref], among others. These molecules are an attractive alternative to peptides for drug design due to their high bioavailability, resistance to proteolysis, and ability to bind competitively at protein binding sites with high affinity by mimicking the conformation of the natural binding partner[ref]. Similarly, incorporating non-natural side chains in peptides may yield increases in binding affinity and specificity as well as protein stability relative to proteins containing only the canonical twenty amino acids[ref].

\subsection{Changing protein sequence is a unique problem in protein modeling}
Unlike continuous, quantitative transformations to a protein structure, such as altering internal coordinates, changes in sequence are discrete and qualitative- Thus one major strategy for design is to consider each amino acid-rotamer pair as a unique state that a given residue position can occupy, and design by selecting low energy states from among those sampled at a given position[ref]. This process is analogous to rotameric sidechain repacking.
One issue with this state-based approach is that residues with more chi angles are intrinsically disfavored: while highly represented among all states, very few rotamers for a many-chi residue will be energetically favorable. As a result, the probability of sampling the low-energy rotamers of these sparse-landscape residues is decreased relative to other residues with many acceptable rotamers[ref?].
Thus, it is a common practice to incorporate reference energies in order to maximize a design protocol's recovery of native protein sequences. These reference energies effectively re-bias the underlying amino acid distribution such that undersampled residues are considered more favorable, thus increasing the number of rotamer states that are energetically acceptable for those residues and allowing them to be designed at a frequency comparable to their natural rate. 

\subsection{An unfolded state energy as a reference energy proxy}
While this statistical weighting works well for canonical amino acid types for which an ideal native residue frequency exists, non-canonical amino acids(NCAAs) and peptidomemetics cannot benefit from a statistically derived reference energy or a sequence recovery benchmark due to a lack of experimental structural data. In addition, many NCAA's have bulky hydrophobic groups with many atoms, which tend to be favored by the Rosetta energy function, causing smaller residues to be underrepresented in design[ref?].
Thus, a new strategy was employed to provide NCAA weights as part of their initial implementation[ref]. A reference energy of the unfolded state was derived via computational mutagenesis in 5-mer fragments for each type of NCAA in Rosetta[ref]. This energy served a similar purpose to the statistical reference energy by representing the baseline solvent and neighboring residue interactions, achieving some balance in designing NCAA residues. NEEDS INSIGHT FROM DOUG

\subsection{Design (canonical and non) requires a reference energy term}
While success has been had in designing canonical peptides for a variety of applications using the Rosetta protein modeling and design suite[ref-rosetta][ref-example][ref-example], development to allow the use of noncanonical amino acids and peptidomemetics is ongoing[ref]. A major obstacle for noncanonical and peptidomemetic design is adjusting for the intrinsic energy of each subunit as a model for the unfolded state of the polymer.


\subsection{Single component unfolded state energies are inadequate}
The unfolded state energies permitted the design and synthesis of NCAA-containing peptides and peptidomimetics (BROOKE/KEVIN REF, PEPTOIDS REF?) but were wanting in several ways
\subsubsection{The unfolded state energy protocol prevents the use of $\beta^3$-residues and other oligomer types.}
The reliance on native, peptidic fragment libraries meant that the insertion of a highly non-canonical backbone as the central residue would be difficult. 
For example, the incorporation of peptoid residues into Rosetta required the mutation of residues on either side of the peptoid residue of interest. 
For $\beta^3$-residues, for which the backbone contains an extra methylene unit, insertion would be impossible; certainly, it would require such an alteration to the native context that the utility of a native fragment library would be lost. 
With this in mind, it appeared appropriate to devise a scheme that did not require evaluating a residue in an incompatible, peptidic context. 
\subsubsection{Reference energies derived from the single-component protocol heavily favored large residues.}
The fragment-mutate-score process invariably resulted in residues with reference energies that contained a relatively small attractive term, simply because few fragments made favorable contacts with the mutated residue. This chemical context meant that each marginal atom possessed by a residue was an additional opportunity to make favorable interactions in the folded state. Thus, large residues like arginine and tryptophan predominate barring truly prohibitive repulsive interactions.
\subsubsection{A more flexible system might permit alternative sets of reference energies.}
With the increasing incorporation of peptidomimetic scaffolds into Rosetta, a flexible reference energy system might be employed to give different energetic descriptions in different chemical contexts. A $\beta^3$-homoleucine residue is chemically different in the context of a 3(14) helix versus a mixed $\alpha$-$\beta$ peptide.
