\section{Demo}
This section demonstates how to do things in this document.
There are a few things that make working collaborativly on a large document easier. 
First, \textit{\textbf{please keep a single sentence per line in the \TeX\ files.}}
This makes using version control easier since a small change doesn't look like an entire paragraph changed.
Second, \texttt{git diff -{}-word-diff} is a really nice tool for gettting differences between text documents rather than source code.

\subsection{Quotations}
Modern editors, like Micrsoft Word, try to infer the begining and ends of of quotes to so that the proper formating of the quotation marks can be used. For example, ``test'' and `test' are correct, "test" and 'test' are not.
They call this feature ``smart quotes.''
Latex requires you to be explicite using \texttt{`{}`word'{}'} (back-tic, back-tic, <word>, single-quote, single-quote) rather than \texttt{"word"} (double-quote, <word>, double-quote).
Smarter editors, like Emacs, do this automagically.

\subsection{Equations}
Here is an example equation that uses the equation enviroment and inline equations.
\begin{quote}
The sum of these weighted energy terms is the score of the pose for that scoring function. 

\begin{equation}
E_{\text{pose}} = \sum_{i}^{n_{\text{res}}} W_{x} E_{x,i} + ... + W_{y} E_{y,i}
\end{equation}
Where $n$ is the total number of residues, $W_{x}$ is the weight on energy term $x$ (eg.\ \texttt{fa\_rep}, \texttt{fa\_solv}, etc.), $E_{x,i}$ is the energy for energy term $x$ for the residue at position $i$. 
The individual weights, $W_x$, on each energy term are fit using a special protocol called \texttt{OptE}.
\end{quote}

\subsection{Tables}
An example table is shown in Table~\ref{supptbl:rot_lib_snpshot_nmeo}.
This also shows how to reference tables and figures.

\begin{table}
  \centering
  \caption{Rotamers produced by the KMC or QMS rotamer library creation protocols for the NPHE peptoid side chain for a preceding-$\omega$, $\phi$, and $\psi$ of \ang{0}, \ang{-90}, \ang{180}}
  \label{supptbl:rot_lib_snpshot_nmeo}
  \begin{tabular}{rrrcrrr}
    \toprule
    \multicolumn{3}{c}{KMC} && \multicolumn{3}{c}{QMS} \\
    \cmidrule{1-3} \cmidrule{5-7}
    Prob & $\chi_1$  & $\chi_2$ && Prob & $\chi_1$  & $\chi_2$ \\
    \midrule
    0.98 & \ang{100}  &  \ang{88}  && 0.71 & \ang{82}  &  \ang{-116} \\
    0.01 & \ang{103}  &  \ang{-39} && 0.29 & \ang{-91} &  \ang{-71}  \\
    0.01 & \ang{-92}  &  \ang{107} && ~ & ~ & ~ \\
    0.00 & \ang{-142} &  \ang{52}  && ~ & ~ & ~ \\
    \bottomrule
  \end{tabular}
\end{table}

\subsection{Bibliography}
There will be two bibliography files: \texttt{auto.bib} and \texttt{byhand.bib}.
The former will contain an auto-generated bibliography file exported from a reference manager.
The later is for references that are difficult to get in to the reference manager (eg.~Gaussian).
If you have any papers you want to add, send them to me and I will add them to my reference manager and update the file.
Here is an example citation\cite{jacak_computational_2012}.
