\section{Methods}
\subsection{Determining the single-body reference energy component}
Acetylated and N-methylamidated protein residues, or acetylated and N-dimethylamidated peptide residues, were constructed in Rosetta. The backbone angles of these residues were scanned to find the lowest total intra-residue energy. With the canonical scoring function, the intra-residue terms were , and , while with mm\_std, the intra-residue terms were . For each set of backbone dihedrals, the backbone was minimized, and side chain was repacked. The reference energy's one body component was taken as the smallest energy thus determined.

METHOD FIGURE

\subsection{Generating per-atom-type reference energies}

Each from the curated Top8000 benchmark set (RICHARDSON REF) was scored with Rosetta using a modified procedure that recorded every inter-atomic score term. These score terms were grouped by atom type and statistics were computed on the scoring distributions to provide potential reference energies. We examined different atom type sets via which to group atoms from the data set. As two limit cases, we examined elemental types, wherein all atoms of the same element are alike, as well as unique atom types, where every atom from every residue type was assigned a different group. Intended to balance generalizability and goodness of fit, we also included the atom type sets used by Rosetta and by CHARMM.(CITE)

TABLE illustrating these atom types. I'm thinking a table at least containing GLY CA, ALA CA, ALA CB, and ALA C as rows. Elemental would group them all as C, MM/Rosetta would distinguish the CA+CB from C, Unique would distinguish all 4.

We considered different methods for obtaining an appropriate measure of centrality for the different distributions produced by these scoring terms, including the mean, median, mode, and Boltzmann-weighted average.
 
Finally, a small number of required atom types were only found in non-canonical amino acids and thus were un-represented in the Top8000 data set. To apply the same procedure to these cases, we mutated chemically similar amino acids, removed any clashes generated from the structure, and then scored. (For example, to obtain parameters for the aliphatic fluorine atom type, we mutated alanines to beta-fluoro-alanine and leucine to various fluoroleucines.)