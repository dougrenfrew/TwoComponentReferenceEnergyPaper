\section{Methods}

\subsection{Rosetta scoring functionality and relevant score functions}
\paragraph{}
Rosetta is dependent on a scoring function to evaluate the fitness of a given molecular structure, both for conformational optimization and for fixed backbone design.
Each score function consists of a series of energy/score terms, which are numerical evaluation metrics which are evaluated on some element of a given protein structure(often called a pose).
These energy terms can be either physical or statistical, and each is designed to be evaluated at some level of organization, such as per-residue evaluation, per-atom evaluation, or per-atom-pair evaluation.
Each energy term in a given scoring function is also assigned a weight value, which is multiplied with the sum of all evaluations of the corresponding energy term.
The product of the energy terms and their weight values is then summed to produce the total score for the protein pose under evaluation.

\begin{quote} 
The Rosetta scoring function can be expressed as follows:

\begin{equation}
E_{\text{pose}} = \sum_{i}^{n_{\text{res}}} W_{x} E_{x,i} + ... + W_{y} E_{y,i}
\end{equation}
Where $n$ is the total number of residues, $W_{x}$ is the weight on energy term $x$ (eg.\ \texttt{fa\_rep}, \texttt{fa\_solv}, etc.), $E_{x,i}$ is the energy for energy term $x$ for the residue at position $i$. 
The individual weights, $W_x$, on each energy term are typically fit using a special protocol called \texttt{OptE}[ref-OptE].
\end{quote}

\paragraph{}
The most common scoring function in Rosetta is $talaris2013$, which was designed for general purpose protein evaluation using canonical amino acids[ref-talaris].
It consists of the score terms $fa\_atr$, $fa\_rep$, $fa\_sol$, $fa\_elec$, $fa\_intra\_rep$, $pro\_close$, $hbond\_sr\_bb$, $hbond\_lr\_bb$, $hbond\_bb\_sc$, $hbond\_sc$, $dslf\_fa13$, $rama$, $omega$, $fa\_dun$, $p\_aa\_pp$, and $ref$.
Full descriptions of these energy terms can be found in [ref-talaris].
%optionally include descriptions of these terms here.

\paragraph{}
In addition to $talaris2013$, there also exists a score function developed for design using noncanonical amino acids and peptidomemetics(protein-like molecules), called $mm\_std$[ref-mm\_std].
The energy terms of this score function are $fa\_atr$, $fa\_rep$, $fa\_sol$, $mm\_lj\_intra\_atr$, $mm\_lj\_intra\_rep$, $mm\_twist$, $pro\_close$, $hbond\_sr\_bb$, $hbond\_lr\_bb$, $hbond\_bb\_sc$, $hbond\_sc$, $dslf\_ss\_dst$, $dslf\_cs\_ang$, $dslf\_ss\_dih$, $dslf\_ca\_dih$, and $unfolded$.
The first difference from $talaris2013$ is the lack of many of the statistical terms, such as $p\_aa\_pp$, the deviation from the phi-psi preferences of the sidechain at each position, which may not be known for a given class of NCAA.
Other major differences include the addition of several molecular mechanics-based score terms, as well as the replacement of the statistically derived reference energy $ref$ with an explicit energetic model of the unfolded state of each residue, called $unfolded$.
This $unfolded$ term serves the same role as $ref$ in the $talaris2013$ score function- to provide a baseline weight for each residue type which corrects for the design biases inherent in Rosetta.


\subsection{Generating per-atom-type two-body energy distributions}
\paragraph{}
The two body portion of our two component reference energy consists of energy values for a set of atoms which are summed to determine the reference energy for any given amino acid or peptidomemetic.
These energy values are obtained from an ensemble of a large number(above order $10^6$ for most atom types) of instances of that atom type in high resolution X-ray protein structures.
These protein structures were obtained from the Top8000 benchmark set of high quality structures[ref-top8000], and was scored with Rosetta using a modified protocol that recorded every inter-atomic energy value.
The two body score terms recorded were $fa\_atr$, $fa\_rep$, $fa\_sol$, $fa\_elec$, $hbond$, and $dslf\_fa13$.
These energy values were grouped by atom type, and statistics were computed on the score distributions to provide expected energies for that atom type in a well-folded protein.
We tested several different sets of atom types to determine which description of the atoms of an amino acid residue would best describe the inter-residue interactions expected for that residue.
As two limit cases, we examined elemental types, wherein all atoms of the same element are alike, as well as unique atom types, where every atom from every residue type was assigned a different group.
Intended to balance generalizability and goodness of fit, we also included the modified CHARMM atom type set used by Rosetta[ref-CHARMM][ref-atom types in rosetta?], as well as molecular mechanics(MM) atom types[ref?].


%Table to illustrate the differences between the atom type sets. 
\begin{table}[!htbp]

\fontsize{9pt}{9pt}
\selectfont

\begin{tabular}{c|lllll}
Residue & Atom Name & Elemental Atom Type & Rosetta Atom Type & MM Atom Type & Unique Atom Type\\
\hline
GLY & Alpha Carbon & C & CAbb & CT1 & GLY/2\\
ALA & Alpha Carbon & C & CAbb & CT1 & ALA/2\\
ALA & Beta Carbon & C & CH3 & CT3 & ALA/5\\
ALA & Carboxyl Carbon & C & CObb & C & ALA/3\\
%Perhaps two more that differentiate Rosetta and MM?
\end{tabular}

\fontsize{10pt}{11pt}
\selectfont
\caption{Examples of common atom types and how they are considered in each of the four atom type sets.}
\label{tab:atypes_example}

\end{table}



\paragraph{}
We considered different methods for obtaining an appropriate measure of centrality for the different distributions produced by these scoring terms, including the mean, median, mode, and Boltzmann-weighted average.
Early testing showed that the distribution median was more robust across different atom types and performed better in sequence recovery trials, and we decided to use the median values in all further testing.


\subsection{Generating two-body energy distributions for NCAA's} 
\paragraph{}
In addition to the atom type energies derived from the top8000 structure benchmark as described above, a small number of mm atom types for NCAA types currently in Rosetta are not found in canonical amino acids, such as the various halogen atoms, and thus were not represented in the Top8000 data set.
To obtain expected energy distributions for these atom types, we selected NCAA's containing those atom types, identified structurally and chemically similar canonical amino acids(e.g. trifluoro-leucine matched to leucine), and mutated all instances of the canonical analog in the top8000 benchmark to the NCAA of interest.
We then resolved clashes produced by the differences between the residues using rotamer repacking and mimimization, and used the score distributions resulting from iteration over all canonical analog positions in the top8000 to generate median energy values for each mm atom type not present in canonical amino acids.
A full list of molecular mechanics atom types for which two body energy distributions were generated this way can be found in table \ref{tab:atypes_all}.


%Probably want to trim this to just NCAA atom types and/or move it to the supplement...
\begin{table}[!htbp]

\fontsize{7pt}{7pt}
\selectfont

%\begin{tabular}{c|lll}
%Atom Type Set & Atom Type & Natural?\\
%\hline
%Elemental & C & yes\\
%Elemental & O & yes\\
%Elemental & H & yes\\
%Elemental & N & yes\\
%Elemental & S & yes\\
%\hline
%Rosetta & HNbb & yes\\
%Rosetta & Ntrp & yes\\
%Rosetta & aroC & yes\\
%Rosetta & ONH2 & yes\\
%Rosetta & Narg & yes\\
%Rosetta & COO & yes\\
%Rosetta & CH1 & yes\\
%Rosetta & CH2 & yes\\
%Rosetta & CH3 & yes\\
%Rosetta & OCbb & yes\\
%Rosetta & NH2O & yes\\
%Rosetta & Nlys & yes\\
%Rosetta & CAbb & yes\\
%Rosetta & S & yes\\
%Rosetta & Nhis & yes\\
%Rosetta & Npro & yes\\
%Rosetta & Haro & yes\\
%Rosetta & CNH2 & yes\\
%Rosetta & OOC & yes\\
%Rosetta & OH & yes\\
%Rosetta & Hpol & yes\\
%Rosetta & CObb & yes\\
%Rosetta & Nbb & yes\\
%Rosetta & Hapo & yes\\
%\hline
%Rosetta & Cl & no\\
%Rosetta & I & no\\
%Rosetta & Br & no\\
%Rosetta & F & no\\
%\hline
\begin{tabular}{c|lll}
Atom Type & Natural?\\
\hline
HS & yes\\
NC2 & yes\\
NR2 & yes\\
HB & yes\\
HC & yes\\
NR1 & yes\\
HA & yes\\
CPT & yes\\
HR1 & yes\\
HR3 & yes\\
NY & yes\\
HP & yes\\
C & yes\\
CC & yes\\
CA & yes\\
H & yes\\
CA & yes\\
O & yes\\
N & yes\\
S & yes\\
NH1 & yes\\
NH3 & yes\\
NH2 & yes\\
OH1 & yes\\
CT1 & yes\\
CT2 & yes\\
CT3 & yes\\
OC & yes\\
CY & yes\\
SM & yes\\
CP1 & yes\\
CP2 & yes\\
CP3 & yes\\
CPH1 & yes\\
CPH2 & yes\\
\hline
BR & no\\
CAP & no\\
CE1 & no\\
CE2 & no\\
CF1 & no\\
CF3 & no\\
CL & no\\
F1 & no\\
F3 & no\\
HE1 & no\\
HE2 & no\\
I & no\\
OE & no\\
\end{tabular}

\fontsize{10pt}{11pt}
\selectfont
\caption{Molecular mechanics atom types for which two body energies were obtained. The current set of atom types allows for useage of all canonical and most noncanonical amino acids currently implemented in Rosetta.}
\label{tab:atypes_all}

\end{table}


\begin{figure}
  \includegraphics[width=\linewidth]{Figures/test.png}
  \includegraphics[width=\linewidth]{Figures/test.png}
  \includegraphics[width=\linewidth]{Figures/test.png}
  \includegraphics[width=\linewidth]{Figures/test.png}
  \caption{Example histogram of two body energy for atom <whatever the atom is>. The median value shown by the green line is the representitive value used to construct the amino acid two body energies. Other statistical measures shown were investigated, but not used due to reduced performance.}
  \label{fig:tbaedist}
\end{figure}


\subsection{Formulation of the two component reference energy}
\paragraph{}
Our split unfolded energy function is composed of two scoring terms in Rosetta, a one body and a two body component.
During scoring, each of these terms is calculated separately for each residue in the protein, and then weighted and summed similar to each other Rosetta energy term.
This combined energy term serves to replace the standard Rosetta reference energy function, known as $ref$.
The formulation of this two component reference energy for a protein is thus:

\begin{equation}
E_{\text{unfolded}} =  W_{1body} \sum_{i}^{n_{\text{res}}} E_{i,1body} +  W_{2body} \sum_{i}^{n_{\text{res}}} E_{i,2body}
\end{equation}

Where $W_{1body}$ is the weight placed on the one body term, $W_{2body}$ is the weight placed on the two body term, $E_{i,1body}$ is the one body component of the reference energy for residue $i$, and $E_{i,2body}$ is the two body component of the reference energy for residue $i$.

\subsection{Calculation of the two body energy term}
\paragraph{}
The two body portion of our two component reference energy consists of a sum of the energies of each constitutent atom.
These atomic energies are calculated from instances of each atom type in the set used appearing in the top8000 data set of high-quality protein crystal structures, as described above.
These values for each Rosetta score term calculated on a per-atom basis are stored in a lookup table, and per-residue two body unfolded energies are built dynamically as needed during runtime from these atom energy lookups.
The two body reference energy for the protein is calculated as follows:

\begin{equation}
E_{\text{2body}} = W_{2body} \sum_{i}^{n_{\text{res}}} \sum_{j}^{l_{\text{scoreterms}}} W_{j} \sum_{k}^{m_{\text{atoms}}} E_{k}
\end{equation}

Where $W_{two-body}$ is the weight value for the two body term, $n$ is the total number of residues, $l$ is the score terms $fa\_atr$, $fa\_rep$, $fa\_sol$, $fa\_elec$, $hbond$, and $fa\_dslf13$ for the $talaris2013$ scoring function and the score terms $fa\_atr$, $fa\_rep$, $fa\_elec$, $hbond$, and $fa\_dslf13$ for the $mm\_std$ scoring function, $W_{j}$ is the weight on score term $j$, $m$ is the number of atoms in the residue $i$, and $E_{k}$ is the energy of atom $k$ in the two body atom energy lookup table.


\subsection{Determining the single-body reference energy component}
\paragraph{}
For the one body component of the reference energy, representing the inherent energy of a given sidechain in an unfolded chain, we used a Rosetta simulation of acetylated and N-methylamidated versions of each monomer residue type to simulate the presence of neighboring residues in a chain.
The backbone phi and psi angles of these residues were iterated in 1 degree bins, with each bin angle minimized using Rosetta's dfpmin minimizer to find local intra-residue energy minima[ref-dfpmin?].
A low energy sidechain conformation was found for each backbone bin using rotamer repacking followed by minimization identical to that of the backbone.
The lowest energy conformation among those sampled was taken as the ideal minimal monomeric conformation for each residue type, and their intra-residue energies recorded as the baseline energy introduced into a protein structure by designing that residue.
This process was carried out using the intra-residue scoring terms for both the $talaris2013$ score function and the $mm\_std$ score function.
For $talaris2013$, the intra-residue terms are $fa\_intra\_rep$, $pro\_close$, $fa\_dun$, $rama$, $omega$, and $p\_aa\_pp$, while for $mm\_std$, the intra-residue terms are $mm\_lj\_intra\_rep$, $mm\_lj\_intra\_atr$, $mm\_twist$, and $pro\_close$.

\paragraph{}
During runtime, the one body reference energy of a protein is determined by summing the values thus recorded for each residue type in it's sequence, which is then weighted and summed with the other score terms to produce the total energy of the protein.
The formulation of the one body term for a given protein is described below.

\begin{equation}
E_{\text{1body}} = W_{1body} \sum_{i}^{n_{\text{res}}} \sum_{k}^{j_{\text{scoreterms}}} W_{k} E_{restype_{i},k}
\end{equation}

Where $W_{1body}$ is the one body energy term weight, $n$ is the total number of residues, $j$ is the set of scoreterms used to compute the one body energy term for the current scorefunction(either $talaris2013$ or $mm\_std$), $W_{k}$ is the weight applied to score term k under the energy function in use, and $E_{restype_{i},k}$ is the one body minimal energy value for the score term $k$ for the residue type of residue $i$ in the protein.


\subsection{Design Benchmarking via Sequence Recovery}
\paragraph{}
To test the utility of our two component reference energy for protein design, we used the Rosetta native sequence recovery benchmark protocol[ref?].
This protocol consists of a collection of 41 high-quality protein structures which are fully redesigned by Rosetta using a given scoring function to test its ability to recapitulate the native amino acid at each position as well as its ability to recapitulate the native amino acid frequencies, both of which indicate it's ability to design ``natural-like'' protein sequences for a given structure.
To aid in analysis, we divided the 20 canonical amino acids into four categories, each of which is affected differently by different portions of the Rosetta scoring process.
These categories were hydrophobic amino acids, consisting of residues valine, isoleucine, leucine, methionine, phenylalanine, glycine, alanine, proline, tryptophan, and tyrosine, polar amino acids, consisting of residues serine, threonine, asparagine, and glutamine, positively charged amino acids, consisting of residues arginine, lysine, and histidine, and negatively charged amino acids, consisting of aspartic acid and glutamic acid.
Traditionally, Rosetta's scoring function performs much better on hydrophobic amino acids, moderately well on polar amino acids, and much worse on positive and negative amino acids.
While no benchmark set exists for the design of non-canonical residue types, performance on the canonical residue types should be indicative of performance on NCAA types as well, as these types are treated identically during runtime.

