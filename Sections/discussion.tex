\section{Discussion}
For every Rosetta energy function,
\begin{quote}
the sum of these weighted energy terms is the score of the pose for that scoring function. 

\begin{equation}
E_{\text{pose}} = \sum_{i}^{n_{\text{res}}} W_{x} E_{x,i} + ... + W_{y} E_{y,i}
\end{equation}
Where $n$ is the total number of residues, $W_{x}$ is the weight on energy term $x$ (eg.\ \texttt{fa\_rep}, \texttt{fa\_solv}, etc.), $E_{x,i}$ is the energy for energy term $x$ for the residue at position $i$. 
The individual weights, $W_x$, on each energy term are fit using a special protocol called \texttt{OptE}.
\end{quote}
We achieved sequence recovery comparable (SUPERIOR?) to the reference energies used by talaris2013.
Encouragingly, our reference energies correspond to a physically meaningful description of the residue in question, while the reference energies talaris2013 uses are merely optimized parameters.
Furthermore, these reference energies permit the design of any non-canonical amino acid in unique chemical contexts; they markedly improve the native sequence recovery possible with the mm\_std scoring function.